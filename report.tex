\documentclass{article}
\usepackage[utf8]{inputenc}
\usepackage{graphicx}
\usepackage{listings}
\usepackage{xcolor}
\usepackage{hyperref}
\usepackage{geometry}
\usepackage{titlesec}
\usepackage{fancyhdr}
\usepackage{parskip}
\usepackage{booktabs}
\usepackage{array}
\usepackage{amsmath}
\usepackage{lipsum}

% Page layout
\geometry{a4paper, margin=1in}
\setlength{\parindent}{0pt}
\setlength{\parskip}{0.8em}

% Colors
\definecolor{codebg}{rgb}{0.95,0.95,0.95}
\definecolor{keyword}{rgb}{0,0,1}
\definecolor{comment}{rgb}{0,0.5,0}

% Listings style
\lstset{
    backgroundcolor=\color{codebg},
    basicstyle=\footnotesize\ttfamily,
    keywordstyle=\color{keyword},
    commentstyle=\color{comment},
    numbers=left,
    numberstyle=\tiny\color{gray},
    frame=single,
    breaklines=true,
    postbreak=\mbox{\textcolor{red}{$\hookrightarrow$}\space},
}

% Custom section formatting
\titleformat{\section}{\Large\bfseries\filcenter}{}{0em}{\vspace{0.5em}}
\titleformat{\subsection}{\large\bfseries}{}{0em}{\vspace{0.3em}}
\titleformat{\subsubsection}{\bfseries}{}{0em}{\vspace{0.2em}}

% Header/Footer
\pagestyle{fancy}
\fancyhf{}
\rfoot{Page \thepage}
\renewcommand{\headrulewidth}{0pt}

\begin{document}

\begin{titlepage}
    \centering
    \vspace*{2cm}
    {\Huge\bfseries Arbitrary Precision Arithmetic Library\par}
    \vspace{1cm}
    {\Large Final Project for CS1023 Course\par}
    \vspace{2cm}
    {\Large Submitted by\par}
    {\Large\bfseries Asit Patel (CS24BTECH11046)\par}
    \vspace{0.5cm}
    \vspace{1cm}
    {\Large Indian Institute of Technology Hyderabad\par}
    \vfill
    {\large \today\par}
\end{titlepage}

\tableofcontents
\newpage

\section{Abstract}
This project implements a comprehensive arbitrary precision arithmetic library that overcomes the limitations of standard data types in programming languages. The library provides:

\begin{itemize}
    \item Unlimited precision integer arithmetic operations
    \item High-precision floating-point calculations
    \item Exact decimal representations without rounding errors
    \item Four fundamental operations: addition, subtraction, multiplication, and division
    \item Usage of command-line interface for command executions.
\end{itemize}

Designed for educational purposes, this implementation demonstrates how basic arithmetic operations can be extended beyond hardware limitations using clever algorithms and proper data representation.

\section{Introduction}
\subsection{The Need for Arbitrary Precision}
In standard programming languages like Java, numerical data types have fixed sizes that limit their range and precision. For example:

\begin{table}[h]
\centering
\begin{tabular}{lll}
\toprule
\textbf{Type} & \textbf{Size} & \textbf{Range} \\
\midrule
int & 32-bit & $-2^{31}$ to $2^{31}-1$ \\
long & 64-bit & $-2^{63}$ to $2^{63}-1$ \\
float & 32-bit & $\pm1.4\times10^{-45}$ to $\pm3.4\times10^{38}$ \\
double & 64-bit & $\pm4.9\times10^{-324}$ to $\pm1.8\times10^{308}$ \\
\bottomrule
\end{tabular}
\caption{Standard Java Data Type Limitations}
\label{tab:types}
\end{table}

These limitations become problematic in applications requiring very high precision like Financial computations, Scientific simulations, Mathematical research and Cryptography.

\subsection{Our Solution}
Our library solves these problems by:
\begin{itemize}
    \item Representing numbers as strings (allowing unlimited length)
    \item Implementing digit-wise arithmetic operations
    \item Providing precise control over decimal places
    \item Handling edge cases and special scenarios
\end{itemize}

\section{Project Structure}
\subsection{File Organization}
The project follows standard Maven directory structure:

\begin{itemize}
    \item \textbf{src/main/java/arbitraryarithmetic/}
    \begin{itemize}
        \item \texttt{AInteger.java} - Core integer operations
        \item \texttt{AFloat.java} - Floating-point implementation
        \item \texttt{MyInfArith.java} - Command-line interface
    \end{itemize}
    
    \item \textbf{src/main/python/}
    \begin{itemize}
        \item \texttt{MyInfArith.py} - Python wrapper script
    \end{itemize}
    
    \item \textbf{pom.xml} - Maven build configuration
    \item \textbf{README.md} - Usage instructions
\end{itemize}

\subsection{Key Components}
\begin{itemize}
    \item \textbf{AInteger Class}:
    \begin{itemize}
        \item Handles arbitrary precision integers by using string representation
        \item Stores numbers as strings with sign flag for positive negative
        \item Implements all basic arithmetic operations
        \item Implements Karatsuba method for multiplication and recursive long division.
    \end{itemize}
    
    \item \textbf{AFloat Class}:
    \begin{itemize}
        \item Manages high-precision decimals
        \item Tracks decimal positions
        \item Automatically aligns decimals for operations
    \end{itemize}
    
    \item \textbf{MyInfArith}:
    \begin{itemize}
        \item Provides user-friendly command-line interface
        \item Validates input and formats output
    \end{itemize}
\end{itemize}

\section{Design and Implementation}
\subsection{Core Algorithms}
\subsubsection{Karatsuba Multiplication}
This efficient multiplication algorithm reduces the complexity from O(n²) to O(n^{1.585}) by recursively breaking the problem into smaller subproblems:

\begin{enumerate}
    \item Split numbers into high and low parts: $x = x_1 \times 10^m + x_0$
    \item Compute three partial products:
    \begin{itemize}
        \item $z_0 = x_0 \times y_0$
        \item $z_1 = (x_1 + x_0) \times (y_1 + y_0)$
        \item $z_2 = x_1 \times y_1$
    \end{itemize}
    \item Combine results: $z_2 \times 10^{2m} + (z_1 - z_2 - z_0) \times 10^m + z_0$
\end{enumerate}

\subsection{Decimal Handling in AFloat}
The \texttt{pre\_process} method ensures proper decimal alignment:

\begin{lstlisting}[language=Java]
public int pre_process(AFloat other) {
    if(no_decimal_1 > no_decimal_2) {
        // Pad other_number with trailing zeros
        other.value += "0".repeat(no_decimal_1 - no_decimal_2);
    } else {
        // Pad this_number with trailing zeros
        this.value += "0".repeat(no_decimal_2 - no_decimal_1);
    }
}
\end{lstlisting}

This ensures numbers like 12.34 and 5.678 become 12.340 and 5.678 before operations. It make the number of decimal digits same by padding zeroes to the end

\section{Testing with sample Inputs}

\subsection{Representative Test Cases}
\begin{itemize}
    \item \textbf{Large Integer Addition}:
    \begin{itemize}
        \item Input: $10^{100} + 1$
        \item Output: $100\ldots001$ (101 digits)
        \item Verifies carry propagation
    \end{itemize}
    
    \item \textbf{Precise Subtraction}:
    \begin{itemize}
        \item Input: $1 - 0.99999999999999999999$
        \item Output: $0.00000000000000000001$
        \item Tests exact decimal handling
    \end{itemize}
    
    \item \textbf{Multiplication}:
    \begin{itemize}
        \item Input: $123456789 \times 987654321$
        \item Output: $121932631112635269$
        \item Validates algorithm correctness
    \end{itemize}
    
    \item \textbf{Division}:
    \begin{itemize}
        \item Input: $1 / 3$
        \item Output: $0.\overline{3}$ (1000 decimal places)
        \item Demonstrates precision control
    \end{itemize}
    
    \item \textbf{Float Addition}:
    \begin{itemize}
        \item Input: $999999999999.999999999999 + 0.000000000001$
        \item Output: $1000000000000.0$
        \item Tests decimal alignment
    \end{itemize}
\end{itemize}

\section{Build and Usage}
\subsection{Building the Project}
\begin{lstlisting}[language=bash]
# Clone repository
git clone https://github.com/your-repo/arbitrary-arithmetic.git
cd arbitrary-arithmetic

# Build with Maven
mvn clean package

# Output JAR will be at:
# target/Final_Project-1.0-SNAPSHOT.jar
\end{lstlisting}

\subsection{Command-line Usage}
Basic syntax:
\begin{lstlisting}[language=bash]
java -jar target/Final_Project-1.0-SNAPSHOT.jar <int/float> <operation> <num1> <num2>
\end{lstlisting}

Examples:
\begin{itemize}
    \item Integer multiplication:
    \begin{lstlisting}[language=bash]
java -jar target/Final_Project-1.0-SNAPSHOT.jar int multiply 123456789 987654321
    \end{lstlisting}
    
    \item Floating-point division:
    \begin{lstlisting}[language=bash]
java -jar target/Final_Project-1.0-SNAPSHOT.jar float divide 22 7
    \end{lstlisting}
\end{itemize}

\section{Limitations and Future Work}
\subsection{Current Limitations}
\begin{itemize}
    \item \textbf{Performance}: Operations slow down significantly with numbers >10,000 digits
    \item \textbf{Memory Usage}: Large numbers consume substantial memory
    \item \textbf{Functionality}: Limited to basic arithmetic operations
    \item \textbf{Precision}: Floating-point division precision capped at 1000 places
\end{itemize}

\subsection{Future Enhancements}
\begin{itemize}
    \item The code could have been made more cleaner and modular for easy understanding and expansion.
    \item Advanced mathematical functions like square-root, logarithmic and exponential functions.
    \item For some of the operations, more efficient methods could have been used. Using Lists for storing the numbers instead of String can result in faster computations.
    \item Support for rational and irrational numbers can also be added in form of new class.
\end{itemize}

\section{Key Learnings}
This project provided invaluable learning experiences in multiple areas:

\subsection{Technical Skills}
\begin{itemize}
    \item Implementation of advanced algorithms like Karatsuba multiplication
    \item String manipulation at scale. Object-oriented design principles. Exception handling and edge cases.
    \item Build system in Java. Use of Maven for project compilation and Jar file generation.
    \item Usage of Python for executing the Java code using Jar.
    \item Using Git and Github for version control.
    \Item Container Image implementation using Docker. Pushing it on Dockerhub and pulling and running it on other devices.
\end{itemize}

\subsection{Problem Solving}
\begin{itemize}
    \item Debugging complex numerical issues
    \item Performance optimization techniques
    \item Trade-off analysis between different approaches
    \item Testing methodologies for numerical code
\end{itemize}

\section{Conclusion}
This project successfully implements arbitrary precision arithmetic using string-based algorithms, overcoming standard data type limitations. 
This library delivers working arbitrary precision arithmetic but has clear performance limits with large numbers. 
While the core operations function correctly, real-world use would require significant optimization. The project provided valuable experience with numerical algorithms and their implementation from scratch.

\end{document}